{\color{gray}\hrule}
\begin{center}
\section{Complexity Analysis}
\textbf{Here we will analyze the running time and complexity of the components of the proposed methodology}
\bigskip
\end{center}
{\color{gray}\hrule}
% \begin{multicols}{2}
\subsection{H3 Indexing and Grid Operations}
% * Complexity of `h3.geo_to_h3` (typically efficient).
% * Complexity of feature aggregation per cell.
\lipsum[1]
\subsection{Mutual Reachibility Graph Construction from H3 Grid Cells}
% * If using `h3.grid_distance` for all pairs: $O(N^2)$ to compute all distances, where N is number of H3 cells with PV. (Mitigation: only consider k-nearest H3 neighbors or neighbors within a fixed H3 distance to create a sparse graph).
% * If sparse graph: $O(N \cdot \text{avg_degree})$.
\lipsum[1-3]

\subsection{Parallel MST Construction}
* Work: $O(N^2)$ in constant dimensions (for WSPD-based EMST, though for `h3.grid_distance` this might differ if not using WSPD directly but still processing edges by weight). More practically, the efficiency comes from not materializing all pairs/edges.
* Depth (Parallelism): Polylogarithmic, e.g., $O(\log^2 N)$ 

\subsection{Dendrogram Construction & Cluster Partitioning}
\subsection{Overall Work, Scalability, and Complexity}
\subsection{Subsection}
\subsection{Subsection}

% \end{multicols}