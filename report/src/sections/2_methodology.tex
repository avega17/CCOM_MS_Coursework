{\color{gray}\hrule}
\begin{center}
\section{Methodology}
\textbf{Here you will describe your:}
\begin{itemize}
    \item Experiments done and pending
    \item Tools, software, code 
    \item Data available (Description, source)
    \item Results (if any)    
\end{itemize} 
Create a subsection for each part described in the methodology. These subsections can be modified based on the work or progress made during the current month. Additionally, you may create separate sections if needed, for example, to distinguish results based on different strategies. This template is flexible and can be adapted for better presentation.

\bigskip
\end{center}
{\color{gray}\hrule}

\begin{multicols}{2}

\subsection{Data and Processing}
\subsubsection{RS Datasets for STF}
    \begin{itemize}
        \item Dataset 1 - Description, source
        \item Dataset 2 - Description, source
    \end{itemize}
\subsubsection{Preprocessing steps}
    

\subsection{Implementation Details}
\subsubsection{Python Libraries and other software}
\begin{itemize}
    \item \textbf{Python 3.10} - programming language
    \item \textbf{PyTorch 2.0} - deep learning framework
    \item \textbf{jupyterlab} - web-based interactive development environment for Jupyter notebooks
    \item \textbf{numpy + xarray} - N-dimensional array processing
    \item \textbf{Geopandas} - Geospatial data manipulation and analysis
    \item \textbf{Rasterio} - Raster data reading and writing
    \item \textbf{GDAL} - Geospatial data abstraction library
    \item \textbf{mamba} - a faster alternative to the conda python package manager
    \item \textbf{rastervision} - a framework for computer vision and deep learning in remote sensing
    \item \textbf{PyTorch Lightning} - a lightweight wrapper for PyTorch to help with training and testing
    \item \textbf{TorchGeo} - a library for deep learning on geospatial data which includes datasets and pretrained models
\end{itemize}

\subsubsection{Hardware and Compute Resources}
\subsubsection{Code Repository}
link placeholder

\subsection{Evaluation}
    \begin{itemize}
        \item Goal(s): Assess accuracy of reconstructed time series 
        \item Metrics:
        \begin{itemize}
            \item RMSE (Root Mean Square Error)
            \item MAE (Mean Absolute Error)
            \item R2 (Coefficient of Determination)
            \item PSNR (Peak Signal-to-Noise Ratio)
            \item Spectral Angle Mapper (SAM)
            \item ERGAS (Relative Geometric Accuracy)
            \item UIQI (Universal Image Quality Index)
            % validate which are useful for STF 
        \end{itemize}
        \item Baseline 
    \end{itemize}

\subsection{Results}

\end{multicols}
\bigskip