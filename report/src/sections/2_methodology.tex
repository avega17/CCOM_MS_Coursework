{\color{gray}\hrule}
\begin{center}
\section{Methodology}
\textbf{Here you will describe your:}
\begin{itemize}
    \item Experiments done and pending
    \item Tools, software, code 
    \item Data available (Description, source)
    \item Results (if any)    
\end{itemize} 
Create a subsection for each part described in the methodology. These subsections can be modified based on the work or progress made during the current month. Additionally, you may create separate sections if needed, for example, to distinguish results based on different strategies. This template is flexible and can be adapted for better presentation.

\bigskip
\end{center}
{\color{gray}\hrule}

\begin{multicols}{2}

\subsection{Datasets and Processing}
\subsubsection{RS imagery and locations of PV arrays}
    \begin{itemize}
        \item Dataset 1 - Description, source
        \item Dataset 2 - Description, source
    \end{itemize}
\subsubsection{RS Datasets for STF}
    \begin{itemize}
        \item Dataset 1 - Description, source
        \item Dataset 2 - Description, source
    \end{itemize}
\subsubsection{Preprocessing steps}
    

\subsection{Implementation Details}
\subsubsection{Python Libraries and other software}
\begin{itemize}
    \item \textit{Python 3.10} - programming language  
    \item \textit{PyTorch 2.0} - deep learning framework  
    \item \textit{jupyterlab} - web-based interactive development environment for Jupyter notebooks
    \item \textit{numpy + xarray} - array processing
    \item \textit{Geopandas} - Geospatial data manipulation and analysis
    \item \textit{duckdb} - an in-process SQL OLAP database management system
    \item \textit{cubo} - a library for working with cuboids (3D arrays) in Python
    \item \textit{openeo} - python client for the OpenEO API
    \item \textit{Rasterio} - Raster data reading and writing
    \item \textit{GDAL} - Geospatial data abstraction library
    \item \textit{mamba} - a faster alternative to the conda package manager
    \item \textit{rastervision} - a framework for computer vision and deep learning in remote sensing
    \item \textit{PyTorch Lightning} - a lightweight wrapper for PyTorch to help with training and testing
    \item \textit{TorchGeo} - a library for deep learning on geospatial data which includes datasets and pretrained models
    \item \textit{super-gradients} - a library for transfer learning and training of SOTA CV models
    \item \textit{data-gradients} - a library designed for computer vision dataset analysis and visualization
    \item \textit{torchmetrics} - a library for computing metrics in PyTorch
\end{itemize}

\subsubsection{Hardware and Compute Resources}

\begin{itemize}
    \item Personal Developer Machine: 16"Macbook Pro 2021, M1 Max, 32GB RAM, PyTorch supported mps accelerator
    \item Google Colab Pro+: \$50/month subscription used to train efficiently train models on cloud GPUs (up to Nvidia A100) and test clustering algorithms on GPU
    \item Google Earth Engine [Future]: cloud-based geospatial analysis platform for large-scale remote sensing data processing 
    \item Microsoft Planetary Computer [Future]: cloud-based geospatial analysis platform for large-scale remote sensing data processing
    \item AWS [Future]: hosts many open-access geospatial datasets including STAC Catalogs 
    
\end{itemize}

\subsubsection{Code Repository}

At the end of the semester, the \href{https://github.com/avega17/CCOM_MS_Spring_2025_EO_PV_research}{private Github repo} where this report and associated code is hosted will be made public.

\subsection{Evaluation}
    \begin{itemize}
        \item Goal(s): Assess accuracy of reconstructed time series 
        \item Metrics (including their normalized versions):
        \begin{itemize}
            \item RMSE (Root Mean Square Error)
            \item MAE (Mean Absolute Error)
            \item MAPE (Mean Absolute Percentage Error)
            \item MBD (Mean Bias Deviation)
            \item R2 (Coefficient of Determination)
            \item PSNR (Peak Signal-to-Noise Ratio)
            % validate which are useful for STF and regression tasks
        \end{itemize}
        \item Baseline(s):
            \begin{itemize}
                \item Persistence model 
                \item Clear sky components
            \end{itemize}
    \end{itemize}

\subsection{Expected Results}

\begin{enumerate}
    \item Subproblem 1: 
        \begin{itemize}
            \item Description of the expected results
            \item Description of the expected results
        \end{itemize}
    \item Subproblem 2:
        \begin{itemize}
            \item Description of the expected results
            \item Description of the expected results
        \end{itemize}
    \item Subproblem 3:
        \begin{itemize}
            \item Description of the expected results
            \item Description of the expected results
        \end{itemize}
\end{enumerate}

\end{multicols}
\bigskip