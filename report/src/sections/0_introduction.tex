\section{Introduction}



\subsection{[Thesis] General Objective}
    % \vspace{0.5cm}
    \textbf{Regional-scale surveys} of distributed large and small scale \textit{PV solar panel installations}, and \textbf{site-level, short-term forecasting} of \textit{solar irradiance} used to estimate local PV power generation.
    This assessment will primarily produce counts of installed systems, estimates of surface area, estimated PV generation potential while monitoring will provide forecasts of local solar irradiance which are used to model estimates of PV energy generation at specific sites.
    % The work developed for this thesis proposal is particularly inspired by and seeks to build upon the work realized, methodologies outlined, research gaps identified, and future work outlined in \cite{robinson_ms_planet_global_renewables_watch_2025}, \cite{Boussif_neurips_day_ahead_solar_forecasting_2023}, \cite{Li_solarcube_solar_forecasting_2024}, \cite{maxar_germany_pv_dataset}, \cite{Hu_solar_array_pitfalls_2022}, \cite{de-Hoog_sota_survey_2020}, \cite{Bansal_ssl_nowcasting_2022}, \cite{Jiang_rooftop_pv_assessment_2022}, \cite{Tremenbert_Kasmi_pyPV_roof_2023}, and \cite{Yu_deep_solar_2018}.

\subsection{Problem Statement and Motivation} 
% Brief description of the problem, objectives, methodology, and main findings. Less than 500 words.
% emphasize the distinction between energy source replacement and badly labeled "energy transition" which has never ocurred historically and instead new energy sources have always been added to the energy mix
The global response to climate change requires our civilization to implement a rapid transition to renewable energy sources in an effort to 
decarbonize the energy sector as much as possible. The fastest growing renewable energy source is by far \textit{photovoltaic} (PV) solar energy [cite]. Growing at over 41\% per year on average since 2009\cite{kruitwagen_global_inventory_pv_units_2021},
PV solar energy has far outpaced other renewable energy sources such as concentrated solar power (CSP), hydroelectric storage, geothermal energy, and, to a lesser extent, wind energy. 
% include figure for rapidly decreasing cost of PV energy and related battery storage technologies
This rapid growth has led to significant progress in international development goals such as the United Nations Sustainable Development Goals (SDG) which, among other things, 
addresses the population level needs for clean and affordable energy, and actions to tackle climate change and its impacts\cite{maxar_germany_pv_dataset}. 

Historically, one of the most effective means to ensure global cooperation and uniform implementation of environmental and energy policies has been through the establishment of multi-lateral 
international agreements such as the Montreal Protocol, the Kyoto Protocol, and, most recently, the Paris Climate Accords signed by (add figure of \% of countries signed and ref) countries. 
This agreement resulted in \textit{non-binding} commitments to reduce greenhouse gas emissions and goals to limit global warming to 1.5 - 2.0 degrees Celsius above pre-industrial levels. 
Notably, 87\% of these Nationally Determined Contributions (NDCs) aim to increase the share of renewable energy in their energy mix with half specifying empirical generation figures\cite{robinson_ms_planet_global_renewables_watch_2025}. 
\textbf{To maintain any semblance of international accountability to these commitments, these NDCs require extensive and accurate global monitoring and validation of any expansion of renewable energy infrastructure.}

To address this \textit{international policy problem} there has been a variety of efforts to assess the progress of each renewable energy source leading to recurring reports, inventories, assessments, and databases at the global, 
regional, and national levels. Many of these efforts prove insufficient due to only providing aggregated summaries or statistical extrapolations at the national level (IRENA, IEA, BP), or being limited to specific regions 
(i.e. primarily Europe, North America, and China). Particularly in the case of PV solar energy, there has been very significant progress over the last decade in the development of remote sensing and deep learning methods to assess 
the distribution of PV solar energy at the site, regional, and global levels (cite the many PV assessment papers).
Most recently, a collaboration between Microsoft, the Nature Conservancy, and Planet Labs\cite{robinson_ms_planet_global_renewables_watch_2025} has produced \textit{``Global Renewables Watch''}, 
a public dataset of industrial, commercial, and utility-scale PV solar energy and wind energy installations firmly establishing the \textbf{feasibility of using high-resolution satellite imagery and deep learning methods 
to assess the distribution of renewable energy infrastructure at the Planetary scale}.

These publications and datasets have primarily focused on large-scale, utility-scale PV solar energy installations (i.e. badly labeled ``solar farms'') which are 
typically located in remote areas with broad land availability and low land use conflicts. However, for some places, land comes at a premium and concentrated large-scale solar energy installations are 
either not economically or politically feasible due to not being the most efficient use of land and other land use conflicts. This leads to countries or regions where distributed, small-scale rooftop or building-integrated 
PV solar energy installations make up a significant portion of the total installed capacity which is not captured in assessment limited to large-scale installations. 
For example, Robinson et al\cite{robinson_ms_planet_global_renewables_watch_2025} and Kruitwagen et al\cite{kruitwagen_global_inventory_pv_units_2021} use a $\ge10KW$ threshold for the solar arrays they are analyzing which excludes most residential installations.  

Additionally, intermittent renewable energy sources bring about a second \textit{technical problem} that also calls for assessments and inventories at a smaller spatial \textbf{and} temporal scale. The intermittency of renewable energy sources such as wind and solar energy means that energy grid operators require accurate short-term forecasts of energy generation to ensure that national electric grid 
supply and demand are balanced. PV solar energy is particularly sensitive to local weather conditions, regional cloud dynamics, and received solar irradiance which can both be monitored using 
remote sensing methods and local weather stations (cite). There has been recent work in the use of geostationary satellites with high temporal resolution to provide short-term forecasts of solar irradiance at the site level\cite{Bansal_ssl_nowcasting_2022}, 
the use of spectral reflectance features in Computer Vision (CV) models for PV solar array detection\cite{He_universal_pv_spectral_index_2024}, and advances in spatio-temporal data fusion methods to improve the temporal resolution of very-high-resolution (VHR) satellite imagery\cite{Tremenbert_Kasmi_pyPV_roof_2023}. 

This project looks to outline the work required to address these interconnected challenges by tackling three core subproblems leveraging remote sensing data and advances in deep learning methods:
    % \vfill\null
    % \columnbreak

\subsubsection{[CCOM6120] Subproblem 1:}
    \textbf{Detection and Segmentation of distributed rooftop PV Systems using Computer Vision baseline models} 
    % \vspace{0.1cm}
    The first fundamental challenge lies in the automated, accurate, and scalable identification and geometric characterization of distributed rooftop PV panels using very-high-resolution (VHR) multispectral satellite imagery. 
    Generating reliable, up-to-date inventories of these small, dispersed assets is crucial for granular PV potential assessments\cite{Pueblas_workflow_rooftop_PV_assessment_sat_img_2023}\cite{Jiang_rooftop_pv_assessment_2022}, infrastructure planning, 
    monitoring deployment rates against policy goals\cite{de-Hoog_sota_survey_2020}, and producing the georeferenced geometry data required for accurate site-level solar irradiance forecasting and PV energy generation estimates\cite{Bansal_ssl_nowcasting_2022}.
    % \vfill\null
    % \columnbreak
    We will use the final project for CCOM6120 as preliminary work on this subproblem. We will explore baseline segmentation model architectures (e.g. UNet, FPN, PAN, SegFormer, etc.) and 
    encoder combinations (e.g. ResNet, EfficientNet, ViT, Swin, etc.) and how they perform on some existing published and open datasets. As part of the project we will develop and establish some baseline data processing pipelines, model training workflows, and evaluation metrics.

\subsubsection{[CCOM6050] Subproblem 2:} 
    \textbf{Hierarchical Spatial Clustering of global PV installations to optimize global dataset generation} 
    Given a large, globally distributed set of Points-of-Interest (PoIs) representing Photovoltaic (PV) solar panel installations (potentially hundreds of thousands to millions), 
    the objective is to efficiently identify and retrieve subsets of relevant satellite imagery (rasters) from open access archives (e.g., STAC catalogs) that maximize the spatial and temporal coverage of the PoIs. 
    Our \textit{Optimization Goal} is to design an algorithmic framework that \textbf{maximizes} the spatial and temporal coverage of the PoIs by the fetched rasters while simultaneously 
    \textbf{minimizing} the number of raster queries and the volume of data downloaded and processed for model training. 

    \textbf{Why it's a problem:} 
    \begin{itemize}
        \item Naive querying strategies (i.e. per POI, or per cluster of POIs) quickly becomes very inefficient for large, globally distributed datasets of PoIs: 
        leads to excessive API calls or HTTP requests, redundant data retrieval, and high processing overhead.
        \item PV installations are naturally spatially clustered: \textit{but} these clusters are not known a priori and can span across arbitrary administrative boundaries or diverse geographic contexts.
        \item The spatial and temporal coverage of the PoIs is not uniform: being able to \textit{dynamically adjust} the size and shape of the query bounding boxes based on the density of PoIs in a given area and the extent of specific raster items is crucial for efficient data retrieval.
    \end{itemize}

    We will use the final project for [CCOM6050] as preliminary work on this subproblem. We will explore the use of data fusion with geospatial context from Overture Maps, the effectiveness of aggregating our 
    preliminary dataset of hundreds of thousands of PoIs using Discrete Global Grid Systems (DGGS) such as Uber's H3, and how we can leverage hierarchical spatial clustering algorithms over our H3 hexagon cells to 
    produce a data structure and search strategy that can be used to efficiently query and retrieve satellite imagery from open access catalogs.  

\subsubsection{[Thesis] Subproblem 3:} 
    \textbf{Solar irradiance forecasting at specific sites} 
    % \vspace{0.5cm}

    The second major challenge is analyzing \textit{high-temporal-resolution surface reflectance} (intra-hour) time series of geostationary satellite imagery to produce accurate short-term forecasts of solar irradiance at specific sites. 
    This is a critical requirement for energy grid operators to ensure that supply and demand are balanced, especially in the case of intermittent renewable energy sources such as wind and solar energy. 

    While beyond the scope of course proyects and this first iteration of this document, the main related works are found in \cite{Bansal_ssl_nowcasting_2022}, \cite{Li_solarcube_solar_forecasting_2024}, and \cite{Boussif_neurips_day_ahead_solar_forecasting_2023}. 
    \hfill