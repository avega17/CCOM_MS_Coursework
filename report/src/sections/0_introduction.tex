\section{Introduction}



\subsection{General Objective}
    \vspace{0.5cm}
    
    \textbf{Regional-scale surveys} of distributed \textit{rooftop PV solar panels}, and \textbf{site-level, short-term forecasting} of \textit{solar irradiance} used to estimate local PV power generation. \\
    \\
    This assessment will primarily produce counts of installed systems, estimates of surface area, estimated PV generation potential while monitoring will provide forecasts of local solar irradiance which are used to model estimates of PV energy generation at specific sites. \\
    \\ 
    The work developed for this thesis proposal is particularly inspired by and seeks to build upon the work realized, methodologies outlined, research gaps identified, and future work outlined in \cite{robinson_ms_planet_global_renewables_watch_2025}, \cite{Boussif_neurips_day_ahead_solar_forecasting_2023}, \cite{Li_solarcube_solar_forecasting_2024}, \cite{maxar_germany_pv_dataset}, \cite{Hu_solar_array_pitfalls_2022}, \cite{de-Hoog_sota_survey_2020}, \cite{Bansal_ssl_nowcasting_2022}, \cite{Jiang_rooftop_pv_assessment_2022}, \cite{Tremenbert_Kasmi_pyPV_roof_2023}, and \cite{Yu_deep_solar_2018}.

\subsection{Specific Objectives}
    \begin{enumerate}
        \item Survey of rooftop Solar Panel \textit{Arrays} via Small Object Detection and Instance Segmentation using \href{https://www.maxar.com/maxar-intelligence/constellation}{very-high-spatial-resolution} (VHR) Multispectral \textit{satellite} imagery (MSI) ($\leq1$ meter/pixel) of installation locations sourced from a \textbf{global inventory of labeled PV arrays} collated from multiple scientific dataset publications [add citations] 
        \item Leverage algorithms that can aid in optimizing our fetching of global satellite imagery from Open Access catalogs, and the use of \textbf{multi-spectral} data to improve the detection of PV arrays.
        \item \textit{Near-term site-level forecasting} of solar \textbf{Global Horizontal  Irradiance} used to estimate PV power generation via data fusion with Spatio-temporal context from:
        \begin{itemize}
            \item Time-series frames from geostationary satellite sensors with high temporal resolution coarse spatial resolution
            \item Historical Solar irradiance (e.g. NREL's National Solar Radiation Database)
            \item Any available PV power generation ground-truth time series
        \end{itemize}
        
    \end{enumerate}
    
\subsection{Research Gaps and Contribution Goals}
    \begin{enumerate}
        \item Measure performance impact of SOTA Computer Vision architectures (e.g. ViT, Swin, CNN-Transformer hybrids) currently lacking in most recent publications
        \item Regional, and global surveys are limited to large-scale farms using medium resolution sensors ($\sim10m$/pixel). On the other hand, studies using VHR aerial imagery usually only have coverage for local, city-scale surveys. We will use \textit{global 30cm yearly basemaps} or open access catalogs from VHR MSI sensors, primarily from Maxar. 
        \item All notable studies exclusively use RGB image bands. Measure impact of use of PV-specific spectral indices\cite{He_universal_pv_spectral_index_2024} and specifically the benefits of including NIR + SWIR bands available in Maxar sensors
        \item Develop a solution that consciously tackles the challenges identified in \cite{Hu_solar_array_pitfalls_2022} for evaluating and performing comparisons of different remote sensing solar array assessment methodologies (distribution drift, test data quality, level of spatial-aggregation, and proprietary data)
    \end{enumerate}
    \hfill
% \lipsum[1]