% \begin{multicols}{2}
\section{Literature Review}

\subsection{[CCOM6050] Data Fusion Reviews}

\subsubsection{2019-RS-DF-Multisource}
\begin{itemize}
    \item Title: ``Multisource and Multitemporal Data Fusion in Remote Sensing: A Comprehensive Review of the State of the Art''
    \item Authors: Pedram Ghamisi; Behnood Rasti; Naoto Yokoya; Qunming Wang; Bernhard Hofle; Lorenzo Bruzzone
    \item Year: 2019
    \item Link/Source: \href{https://doi.org/10.1109/MGRS.2018.2890023}{10.1109/MGRS.2018.2890023}
    \item Main objective:
    \item Methodology used:
    \item What I got out of the paper:
        \begin{itemize}
            \item “bird’s eye view” of contributions in:
                \begin{itemize}
                    \item pansharpening and resolution enhancement
                    \item multitemporal data fusion
                \end{itemize}
            \item Traditionally in Remote Sensing (RS), there are four dimensions that provide information:
                \begin{itemize}
                    \item Spatial: 2D (x,y) coordinates
                    \item Temporal : 1D (t) time
                    \item Spectral: 1D ($\lambda$) wavelength 
                    \item Radiometric: 1D (r) numerical precision in a pixel's radiance/reflectance/intensity values
                \end{itemize}
            \item RS measurement process is explored via describing the four components of a "physical model":
                \begin{itemize}
                    \item Scene model: defines the subject of interest
                    \item Atmosphere model: defines the transform of Electromagnetic Spectrum (EMS) from surface to sensor
                    \item Sensor model: defines the measurement process (e.g. signal-to-noise ratio (SNR), sweep time, nadir angle, number of bands, etc.)
                    \item Image model: defines the sampling process (e.g. pixel size, spatial resolution, etc.)
                \end{itemize}
            \item \textbf{``All data fusion methods attempt to overcome the above measurement and sampling processes''} \\ 
            ``Understanding these [differences in] measurement and sampling processes is\dots key to characterizing methods of data fusion'' 
            $\implies$ can be approached as a reconstruction problem\dots\\
            \item Spatio-Spectral data fusion
                \begin{itemize}
                    \item Pansharpening: a specific case of spatio-spectral fusion where a high-resolution panchromatic image is fused with a lower-resolution multispectral image. \\
                    This fusion process attempts to preserve the spectral information of the multispectral image while enhancing the spatial resolution for \textit{all} bands. 
                    \item Most methods can be categorized into: 
                        \begin{enumerate}
                            \item Component substitution: spectrally transforms MS data into a new feature space to separate spatial and spectral information, \\ 
                            then substitutes the spatial component is substituted for the high-resolution panchromatic image while using histogram matching to adjust the PAN intensity to match the original spectral information.
                                \begin{itemize}
                                    \item Pros: \textit{high spatial fidelity}; \textit{low computational cost}; \textit{robust against misregistration errors (i.e. spatial misalignment)}
                                    \item Cons: \textit{suffer from global spectral distortion}; \textit{lower spectral fidelity}
                                    \item Methods include:
                                    \begin{enumerate}
                                        \item \textbf{Intensity-Hue-Saturation (IHS)}
                                        \item Principal Component Analysis (PCA)
                                        \item Gram-Schmidt (GS)
                                    \end{enumerate}
                                \end{itemize}
                            \item Multiresolution analysis: extracts spatial high-frequency components from the panchromatic image and injects with coefficients into the low-resolution multispectral image.
                                \begin{itemize}
                                    \item Pros: Spectral consistency (e.g. no spectral loss)
                                    \item Cons: \textit{Higher computational complexity and cost}
                                    \item Methods include:
                                    \begin{enumerate}
                                        \item Box filtering
                                        \item Gaussian filtering
                                        \item Bilateral filtering
                                        \item Wavelet transform
                                        \item Curvelet transform
                                    \end{enumerate}
                                \end{itemize}
                            \item Geostatistical analysis: can preserve the spectral properties of the original coarse image while enhancing the spatial resolution.
                                \begin{itemize}
                                    \item Pros: 
                                    \item Cons: 
                                    \item Methods include: several types of \textit{kriging} (spatial interpolation) which are used to estimate unmeasured values based on neighboring values
                                \end{itemize}
                            \item Subspace representation: uses a subspace spanned by a set of basis vectors to analyze intrinsic spectral characteristics
                                \begin{itemize}
                                    \item Pros: 
                                    \item Cons: \textit{High computational cost}; \textit{can introduce spectral artifacts}
                                    \item Methods include: Bayesian analysis, Matrix Factorization, Spectral Unmixing
                                \end{itemize}
                            \item Sparse representation: captures the spectral signatures of materials in an image patch using a sparse representation with a few basis vectors
                                \begin{itemize}
                                    \item Pros: \textit{Efficient storage and processing}; \textit{high spectral fidelity}; \textit{sparse representation can be used in downstream CV tasks}
                                    \item Cons: \textit{Targeted for HSI-MSI fusion}; \textit{very high computational cost}
                                    \item Methods include: Hierarchical Pyramid models; extraction of spectral signatures; 
                                \end{itemize}
                        \end{enumerate}
                    \item Some quantitative evaluation metrics:
                        \begin{enumerate}
                            \item Peak Signar-to-Noise Ratio (PSNR): evaluates the quality of a reconstructed image by comparing the maximum possible power of a signal to the power of corrupting noise
                            \item Spectral Angle Mapper (SAM): determines the similarity between a transformed image spectrum and a reference spectrum by calculating the angle between their vector representations in the spectral space
                            \item ERGAS (\textit{erreur relative globale adimensionnelle de synthèse}): measures image quality in terms of the per-band normalized mean error between the fused image and the reference image
                            \item Q\textsuperscript{2n}: a global reconstruction quality index 
                        \end{enumerate}
                \end{itemize}
            \item Spatio-temporal data fusion
                \begin{itemize}
                    \item ``a technique to blend fine spatial resolution, but coarse temporal resolution data with fine temporal resolution, but coarse spatial resolution data''
                    \item \textbf{Most methods are based on the strong assumption of no abrupt changes in Land Cover and Land Use (LCLU) across time} (ok for our work if we focus on rooftops)
                    \item Methods mentioned:
                    \begin{enumerate}
                        \item STARFM (Spatial and Temporal Adaptive Reflectance Fusion Model)
                        \item ESTARFM (Enhanced STARFM)
                    \end{enumerate}
                    \item Common sensor pairings: 
                        \begin{enumerate}
                            \item MODIS (~daily; 0.25km-1km) + Landsat (~16 days; 30m)
                            \item MODIS + Sentinel-2 (~5 days; 10m)
                            \item GOES-R (5-60 min; 0.25-2km) + ?? 
                        \end{enumerate}
                    
                \end{itemize}
        \end{itemize}
    \item Key findings or contributions for our topic:
        \begin{itemize}
            \item Outlines the centrality of Coherence for Data Fusion with RS data: \\ 
            
            ``Data fusion is made possible because each dataset to be fused represents a different view of the same real world defined in space and time (generalized by the scene model), 
            with each view having its own measurable properties, measurement processes, and sampling processes.'' \\ 
            \\ 
            $\implies$ ``one should expect some level of \textbf{coherence} between 
            the real world (the source) and the multiple datasets (the observations)''
            % ``The real world is spatially correlated, at least at some scale and this phenomenon has been used in many algorithms''
            \item \textbf{Spatio-temporal fusion} was identified as the relevant type of fusion for our work: \\
            \\
            ``A large focus of attention currently is on the specific problem that arises from the trade-off in remote sensing between spatial resolution and temporal frequency;
            in particular the fusion of coarse-spatial/fine-temporal-resolution with fine-spatial/coarse-temporal-resolution space-time datasets''
            \item For an individual remote sensing platform, ``there always exists a trade-off between spatial resolution and temporal resolution (revisit time)''. This can be worked around with a constellation of multiple sensors. 
            \item A remaining major issue is how to conduct fair comparisons of performance and accuracy of different methods.
        \end{itemize}
\end{itemize}

\subsubsection{2024-DL-Data-Fusion}
\begin{itemize}
    \item Title: ``A Comprehensive Review on Deep Learning-Based Data Fusion'''
    \item Authors: Mazhar Hussain; Mattias O' Nils; Jan Lundgren; Seyed Jalaleddin Mousavirad
    \item Year: 2024
    \item Link/Source: \href{https://doi.org/10.1109/ACCESS.2024.3508271}{10.1109/ACCESS.2024.3508271} 
    \item Main objective: 
    \item Methodology used:
    \item Key findings or contributions for our topic: 
\end{itemize}

\subsection{[CCOM6050] Data Fusion Datasets}

\begin{itemize}
    \item Title: 
    \item Authors:
    \item Year:
    \item Link/Source:
    \item Main objective of the paper:
    \item Methodology used:
    \item Number of annotations:
    \item Date range for annotations:
    \item Annotation locations:
    \item Key findings or contributions:
\end{itemize}

\subsection{PV imagery and locations Datasets}

\subsubsection{2023-SDG-Maxar-PV-dataset}

\begin{itemize}
    \item Title: ``A solar panel dataset of very high resolution satellite imagery to support the Sustainable Development Goals''
    \item Authors: Cecilia Clark [ex-Maxar]; Fabio Pacifici[Maxar]
    \item Year: 2023
    \item Link/Source: \href{https://doi.org/10.1038/s41597-023-02539-8}{10.1038/s41597-023-02539-8} | 
    \href{https://resources.maxar.com/geospatial-foundation/15-cm-hd-and-30-cm-view-ready-solar-panels-germany}{imagery} |
    \href{https://figshare.com/articles/dataset/Solar_Panel_Object_Labels/22081091}{annotations}
    \item Main objective of the paper: 
    To provide a VHR satellite imagery dataset of annotated, primarily residential, solar panels to support UN's Sustainable Development Goals, and further improve solar panel detection models.
    \item Methodology used: \\
    Obtained 31 cm resolution satellite imagery and applied proprietary HD processing to generate 15.5 cm resolution imagery used for panel detection.
    Labels created with Object Detection model (YOLT) and validated by human annotators.
    \item Number of annotations: 2,542
    \item Date range for annotations: 2020/09/18
    \item Annotation locations: Southern Germany 
    \item Key findings or contributions: \\
    The dataset is designed to support small object detection and focuses on annotated, primarily residential, solar panels. 
    Includes paired native resolution (31 cm) and HD (15.5 cm) satellite imagery. 
\end{itemize}

\subsubsection{2021-global-PV-inventory}

\begin{itemize}
    \item Title: ``A global inventory of photovoltaic solar energy generating units''
    \item Authors: L. Kruitwagen; K. T. Story; J. Friedrich; L. Byers; S. Skillman; C. Hepburn 
    \item Year: 2021
    \item Link/Source: \href{https://doi.org/10.1038/s41586-021-03957-7}{10.1038/s41586-021-03957-7} | 
    \href{https://zenodo.org/records/5005868}{PV Labels} | 
    \href{https://github.com/Lkruitwagen/solar-pv-global-inventory}{Code Repo}
    \item Main objective of the paper: To provide a global inventory of commercial, industrial, and utility-scale PV installations ($\geq 10kW$ nominal generation capacity)
    \item Methodology used: Machine Learning pipeline (series of CNN + RNN models), OpenStreetMap annotations, heuristic filters, negative sampling (non solar panel objects)
    \item Number of annotations: 68,661 facilities [36,882 (Sentinel-2), 38,541 (SPOT)]
    \item Date range for annotations: 2016/06/01 - 2018/09/30
    \item Annotation locations: Global (131 countries)  
    \item Key findings or contributions: The dataset expands previous publicly available data by $> 4x$. Provides an estimate of global installed generating capacity: $423 (~\pm75) GW$. Mentions importance of spectral signature of PV panels and extraction of spectral features. Provides implementation repo. ``The pipeline has two stages: an initial global search designed to maximize installation recall, followed by a process to remove false positives and estimate installation dates.''
\end{itemize}

\subsubsection{2020-UK-Solar-PV}

\begin{itemize}
    \item Title: ``A harmonised, high-coverage, open dataset of solar photovoltaic installations in the UK''
    \item Authors: Dan Stowell; Jack Kelly; Damien Tanner; Jamie Taylor; Ethan Jones; James Geddes; Ed Chalstrey
    \item Year: 2020
    \item Link/Source: \href{https://doi.org/10.1038/s41597-020-00739-0}{10.1038/s41597-020-00739-0} | 
    \href{https://zenodo.org/records/4059881}{PV Labels} |
    \href{https://github.com/openclimatefix/solar-power-mapping-data}{Code Repo}
    \item Main objective of the paper: To create an open geographic data source for solar PV, suitable for intra-nation (UK in this case) analysis using machine vision and PV forecasting. 
    \item Methodology used: OpenStreetMap, Crowdsourcing, deduplication via spatial clustering
    \item Number of annotations: over 260,000 (over 255K separate installations; 1067 large solar farms )
    \item Date range for annotations: Data was collected up to September 2020 
    \item Annotation locations: United Kingdom
    \item Key findings or contributions: It includes a large number of small-scale domestic installations, which were typically poorly documented. 
    Provides detailed metadata and location geometries. 
    Discusses challenges of data reconciliation and deduplication which is relevant to our work that will merge multiple datasets.
    Includes a GUI for data visualization and validation which will likely be relevant or useful for our work
\end{itemize}

\subsubsection{2023-US-large-PV-EIA}

\begin{itemize}
    \item Title: ``Georectified polygon database of ground-mounted large-scale solar photovoltaic sites in the United States''
    \item Authors: K. Sydny Fujita; Zachary H. Ancona; Louisa A. Kramer; Mary Straka; Tandie E. Gautreau; Dana Robson; Chris Garrity; Ben Hoen; Jay E. Diffendorfer 
    \item Year: 2023
    \item Link/Source: \\
    \href{https://doi.org/10.1038/s41597-023-02644-8}{10.1038/s41597-023-02644-8} | 
    \href{https://www.sciencebase.gov/catalog/item/6671c479d34e84915adb7536}{PV annotations} 
    \item Main objective of the paper: To develop a comprehensive, publicly available georectified dataset of ground-mounted 
    large-scale solar photovoltaic (LSPV) facilities in the US. 
    \item Methodology used: Georectification of PV facility coordinates from EIA data. 
    Used high-resolution aerial imagery to validate facilities and digitize location polygon. 
    QA/QC by human annotators. Facility metadata was attached to geometries. 
    \item Number of annotations: 3,699 ground-mounted LSPV facilities
    \item Date range for annotations: Vector geometries with no attached imagery. Facilities became operational between 2018 and 2023. 
    \item Annotation locations: Continental USA
    \item Key findings or contributions: N/A
\end{itemize}

% \begin{itemize}
%     \item Title: 
%     \item Authors:
%     \item Year:
%     \item Link/Source:
%     \item Main objective of the paper:
%     \item Methodology used:
%     \item Number of annotations:
%     \item Date range for annotations:
%     \item Annotation locations:
%     \item Key findings or contributions:
% \end{itemize}


% \subsection{Year-Short title [cite]}
% For Each Paper make a subsetion: containing:
% \begin{itemize}
%     \item Title
%     \item Authors
%     \item Year
%     \item Link/Source
%     \item Main objective of the paper: 
%     \item Methodology used: (theoretical, experimental, simulation, literature review, etc.)
%     \item Key findings or contributions
% \end{itemize}
% \lipsum[1][10]


\subsection{2024-LSTM, Sepsis, PPG\cite{alvarez2024lstm}}
Example of the subsection title: ``2024 LSTM, Sepsis, PPG'', which is a shortened version of: ``2024 LSTM Model for Sepsis Detection and Classification Using PPG Signals.''

\azul{In this report, each paper must fit within one page! I know that reviewing a paper could take more space, but for that, use a separate document.}

 I intentionally use \azul{BLUE} to capture attention, \rojo{RED} for negative or unexpected results/data, and \verd{GREEN} for positive ones.

\subsection{20xx ASDF}
A monthly report usually includes three news articles. Two is acceptable, four is good, and more than six is excellent.
% \end{multicols}