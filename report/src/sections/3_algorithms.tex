{\color{gray}\hrule}
\begin{center}
\section{[CCOM6050] Adapting Spatial Hierarchical Clustering Algorithms to our application}
\textbf{Here we will go into detail on our algorithm design and approach for optimization of STAC queries when generating our dataset.}
\bigskip
\end{center}
{\color{gray}\hrule}

\begin{multicols}{2}

\subsection{Data Ingestion and Pre-processing}

% * Sources of PV solar panel label datasets (cite your aggregated sources).
% * Cleaning, geometric validation, and deduplication techniques.
% * *(Potential for Map Point Reduction (MPR) inspired by Oje et al., 2025, if initial PV data is excessively dense/redundant before H3 aggregation).*

\subsection{H3 Indexing and Spatial Aggregation}
% * Algorithm for assigning PV labels to H3 cells at a chosen resolution.
%     * Discussion on selecting H3 resolution(s).
% * Derivation of H3 cell-level features (e.g., PV count, total PV area, installation date statistics).
% * *(Optional: Augmentation with Overture Maps Land Cover or other STAC-derived features per H3 cell).*

\subsection{Graph Construction from H3 Grid Cells}
% * Nodes: H3 cells containing PV installations.
% * Edge Weight Definition:
%     * Primary: `h3.grid_distance(cell_a, cell_b)` to define topological proximity.
%     * *(Discussion of potential hybrid approaches if feature similarity is incorporated).*
% * Graph representation (e.g., adjacency list for sparse graph if only considering neighbors within a certain H3 distance).

\subsection{Minimum Spanning Tree (MST) Construction}
% * Algorithm: Parallel GeoFilterKruskal (GFK) with MemoGFK optimization (Wang et al., 2021).
% * Justification: Scalability for large numbers of H3 cells, memory efficiency.
% * *(If HDBSCAN*-like density focus is explored: discuss core distance definition for H3 cells and Wang et al.'s specialized well-separation).*

\subsection{Generating a Hierarchy of Clusters: Dendogram Construction and H3 multi-resolution}
% * Algorithm(s):
%     * RC-Tree Tracing (RCTT) (Dhulipala et al., 2024) for CPU-based parallelism.
%     * PANDORA (Sao et al., 2024) for GPU-accelerated computation.
% * Justification: Work-optimality (or near-optimality), practical speedups, robustness to dendrogram skew.

\subsection{Optimized STAC Querying}
% * Algorithm for generating STAC query parameters (bounding boxes, geometries, time ranges) based on the identified H3 cell cluster footprints.
% * Strategy for minimizing redundant queries (e.g., merging overlapping cluster query geometries).

% \lipsum[1][10-15]

% \subsubsection{Statistical STF algorithm candidates}
% (add citations)

% \begin{itemize}
%     \item Weight Function-based methods
%         \begin{enumerate}
%             \item STARFM (Spatial and Temporal Adaptive Reflectance Fusion Model): \\
%             One of the earliest and most widely used methods. It assumes land cover change is minimal between image acquisition dates  
%             (ok for our application) and uses spectral similarity and spatial distance to weight contributions from known fine-resolution 
%             pixels to predict reflectance for dates where only coarse imagery is available. 
%             \item ESTARFM (Enhanced STARFM): \\
%             An extension of STARFM designed to perform better in heterogeneous landscapes by considering spectral unmixing concepts implicitly 
%             and using conversion coefficients derived from two fine/coarse image pairs before and after the desired prediction date. 
%         \end{enumerate}
%     \item Unmixing-based methods
%         \begin{enumerate}
%             \item STDFA (Spatio-Temporal Data Fusion Approach): \\
%             Combines spectral unmixing and spatial interpolation to predict fine-resolution reflectance. 
%         \end{enumerate}
%     \item Regression-based methods
% \end{itemize}

% \subsubsection{Deep Learning STF algorithm candidates}
% \begin{itemize}
%     \item CNN-based methods
%     \item GAN-based methods
% \end{itemize}

% \subsection{Implementation}
% (add links to colab jupyter notebooks) \\
% (add essential code excerpts to highlight algorithm details)

% \subsection{Complexity Analysis}

% \begin{itemize}
%     \item CV factors: image size, number of bands, number of samples
%     \item DL factors: number of layers, batch size, number of epochs, fp weight size
%     \item non-DL factors: window size, other parameters 
%     \item Big O notation
%     \item Running time 
%     \item Memory usage
% \end{itemize}

\clearpage

\end{multicols}

\clearpage

