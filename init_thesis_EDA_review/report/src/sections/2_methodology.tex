{\color{gray}\hrule}
\begin{center}
\section{Methodology}
% \textbf{Here you will describe your:}
% \begin{itemize}
%     \item Experiments done and pending
%     \item Tools, software, code 
%     \item Data available (Description, source)
%     \item Results (if any)    
% \end{itemize} 
% Create a subsection for each part described in the methodology. These subsections can be modified based on the work or progress made during the current month. Additionally, you may create separate sections if needed, for example, to distinguish results based on different strategies. This template is flexible and can be adapted for better presentation.

\bigskip
\end{center}
{\color{gray}\hrule}

\subsection{Datasets of RS Imagery and Locations of PV Arrays}
    Datasets in bold were used in the data preparation and analysis described in the next section. 
    \begin{table*}[htbp]
        \centering
        \scriptsize
        \caption{Summary of identified datasets for PV array segmentation and detection.}
        \begin{tabularx}{.95\textwidth}{|p{0.42\textwidth}|p{0.1\textwidth}|p{0.1\textwidth}|p{0.23\textwidth}|}  \hline
            \textbf{Dataset Title} & \textbf{Author, Year} & \textbf{DOI Links} & \textbf{Label Information} \\
                \hline
                A 10-m national-scale map of ground-mounted photovoltaic power stations in China of 2020 & Feng et al., 2024 & \href{https://doi.org/10.1038/s41597-024-02994-x}{Paper}\linebreak \href{https://doi.org/10.57760/sciencedb.o00121.00001}{Dataset} & 5K Positive samples and 5K negative samples in China, spring 2020 \\
                \hline
                GloSoFarID: Global multispectral dataset for Solar Farm ID & Yang, 2024 & \href{https://doi.org/10.48550/arXiv.2404.05180}{Paper}\linebreak \href{https://github.com/yzyly1992/GloSoFarID/tree/main/data_coordinates}{Dataset} & 6,793 PV samples across 3 years \\
                \hline
                Vectorized solar PV installation dataset across China & Liu et al., 2024 & \href{https://doi.org/10.1038/s41597-024-04356-z}{Paper}\linebreak \href{https://github.com/qingfengxitu/ChinaPV}{Dataset} & 3,356 PV labels \\
                \hline
                \textbf{A solar panel dataset of very high resolution satellite imagery} & Clark et al., 2023 & \href{https://doi.org/10.1038/s41597-023-02539-8}{Paper}\linebreak \href{https://doi.org/10.6084/m9.figshare.22081091.v3}{Dataset} & 2,542 object labels \\
                \hline
                Crowdsourced aerial images with annotated solar PV arrays & Kasmi, 2023 & \href{https://doi.org/10.1038/s41597-023-01951-4}{Paper}\linebreak \href{https://doi.org/10.5281/zenodo.6865878}{Dataset} & $>$28K PV points\linebreak 13K$+$ segmentation masks \\
                \hline
                Georectified polygon database of ground-mounted large-scale solar PV sites in US & Sydny, 2023 & \href{https://doi.org/10.1038/s41597-023-02644-8}{Paper}\linebreak \href{https://www.sciencebase.gov/catalog/item/6671c479d34e84915adb7536}{Dataset} & 4,186 data points \\
                \hline
                \textbf{An Artificial Intelligence Dataset for Solar Energy Locations in India} & Ortiz, 2022 & \href{https://doi.org/10.1038/s41597-022-01499-9}{Paper}\linebreak \href{https://github.com/microsoft/solar-farms-mapping/blob/main/data/solar_farms_india_2021_merged_simplified.geojson}{Dataset} & 117 geo-referenced solar installations \\
                \hline
                \textbf{A global inventory of photovoltaic solar energy generating units} & Kruitwagen et al., 2021 & \href{https://doi.org/10.1038/s41586-021-03957-7}{Paper}\linebreak \href{https://doi.org/10.5281/zenodo.5005867}{Dataset} & 50,426 training samples\linebreak 68,661 predicted labels \\
                \hline
                Multi-resolution dataset for PV panel segmentation & Jiang, 2021 & \href{https://doi.org/10.5194/essd-13-5389-2021}{Paper}\linebreak \href{https://doi.org/10.5281/zenodo.5171712}{Dataset} & 3,716 PV data points \\
                \hline
                \textbf{A harmonised, high-coverage, open dataset of solar PV installations in the UK} & Stowell et al., 2020 & \href{https://doi.org/10.1038/s41597-020-00739-0}{Paper}\linebreak \href{https://zenodo.org/records/4059881}{Dataset} & 265,418 data points \\
                \hline
                \textbf{Harmonised global datasets of wind and solar farm locations and power} & Dunnett et al., 2020 & \href{https://doi.org/10.1038/s41597-020-0469-8}{Paper}\linebreak \href{https://doi.org/10.6084/m9.figshare.11310269.v6}{Dataset} & 35,272 PV installations \\
                \hline
                \textbf{Distributed solar photovoltaic array location and extent dataset} & Bradbury, 2016 & \href{https://doi.org/10.1038/sdata.2016.106}{Paper}\linebreak \href{https://doi.org/10.6084/m9.figshare.3385780.v4}{Dataset} & 19,433 PV modules in 4 California cities \\
                \hline
        \end{tabularx}
        \label{tab:pv_datasets}
    \end{table*}

\begin{multicols}{2}

\subsection{Python Libraries}
\begin{itemize}
    \item \textit{Python 3.10} - programming language  
    \item \textit{PyTorch 2.0} - deep learning framework 
    \item \textit{jupyterlab} - web-based interactive development environment for Jupyter notebooks
    \item \textit{numpy + xarray} - array processing
    \item \textit{Geopandas} - Geospatial data manipulation and analysis
    \item \textit{duckdb} - an in-process SQL OLAP database management system
    \item \textit{cubo} - a library for working with cuboids (3D arrays) in Python
    \item \textit{stac-geoparquet} - Convert STAC items between JSON, GeoParquet, pgstac, and Delta Lake; allows users to access a large number of STAC items in bulk without making repeated HTTP requests
    \item \textit{openeo} - python client for the OpenEO API
    \item \textit{Rasterio} - Raster data reading and writing
    \item \textit{GDAL} - Geospatial data abstraction library
    \item \textit{mamba} - a faster alternative to the conda package manager
    \item \textit{rastervision} - a framework for computer vision and deep learning in remote sensing
    \item \textit{PyTorch Lightning} - a lightweight wrapper for PyTorch to accelerate model training and testing
    \item \textit{TorchGeo} - a library for deep learning on geospatial data which includes datasets and pretrained models
    \item \textit{segmentation-models-pytorch} - a library for pre-trained semantic segmentation models and vision encoders in PyTorch
    % \item \textit{super-gradients} - a library for transfer learning and training of SOTA CV models
    % \item \textit{data-gradients} - a library designed for computer vision dataset analysis and visualization
    \item \textit{torchmetrics} - a library for computing model evaluation metrics in PyTorch
\end{itemize}

\subsection{Hardware and Compute Resources}

\begin{itemize}
    \item Personal Developer Machine: 16'' Macbook Pro 2021, M1 Max, 32GB RAM, PyTorch supported mps accelerator
    \item Google Colab Pro+: \$50/month subscription used to efficiently train models on cloud GPUs (up to Nvidia A100) and test clustering algorithms on GPU
    \item Google Earth Engine [Future]: cloud-based geospatial analysis platform for large-scale remote sensing data processing 
    \item Microsoft Planetary Computer [Future]: cloud-based geospatial analysis platform for large-scale remote sensing data processing
    \item AWS [Future]: hosts many open-access geospatial datasets including STAC Catalogs and offers cheap cloud object storage
    
\end{itemize}

\subsection{Code Repository}

As of writing, the \href{https://github.com/avega17/CCOM_MS_Spring_2025_EO_PV_research}{Github repo} where this report and associated code is hosted is public.

% \subsection{Evaluation}
    % \begin{itemize}
    %     \item Goal(s): Assess accuracy of reconstructed time series 
    %     \item Metrics (including their normalized versions):
    %     \begin{itemize}
    %         \item RMSE (Root Mean Square Error)
    %         \item MAE (Mean Absolute Error)
    %         \item MAPE (Mean Absolute Percentage Error)
    %         \item MBD (Mean Bias Deviation)
    %         \item R2 (Coefficient of Determination)
    %         \item PSNR (Peak Signal-to-Noise Ratio)
    %         % validate which are useful for STF and regression tasks
    %     \end{itemize}
    %     \item Baseline(s):
    %         \begin{itemize}
    %             \item Persistence model 
    %             \item Clear sky components
    %         \end{itemize}
    % \end{itemize}
    % \subsubsection{Computer Vision Models}
    %     \begin{itemize}
    %         \item Dice Loss
    %         \item mean Intersection over Union (mIoU): 
    %     \end{itemize}

% \subsection{Expected Results}

% \begin{enumerate}
%     \item Subproblem 1: 
%         \begin{itemize}
%             \item Description of the expected results
%             \item Description of the expected results
%         \end{itemize}
%     \item Subproblem 2:
%         \begin{itemize}
%             \item Description of the expected results
%             \item Description of the expected results
%         \end{itemize}
%     \item Subproblem 3:
%         \begin{itemize}
%             \item Description of the expected results
%             \item Description of the expected results
%         \end{itemize}
% \end{enumerate}

\end{multicols}
\bigskip