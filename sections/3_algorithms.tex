{\color{gray}\hrule}
\begin{center}
\section{[CCOM6050] Data Fusion Algorithms}
\textbf{Here we will go into more detail and implement one traditional data fusion algorithm and another using Deep Learning methodology.}
\bigskip
\end{center}
{\color{gray}\hrule}

\begin{multicols}{2}

\subsection{Algorithm Design}
% \lipsum[1][10-15]

\subsubsection{Statistical STF algorithm candidates}
(add citations)

\begin{itemize}
    \item Weight Function-based methods
        \begin{enumerate}
            \item STARFM (Spatial and Temporal Adaptive Reflectance Fusion Model): \\
            One of the earliest and most widely used methods. It assumes land cover change is minimal between image acquisition dates  
            (ok for our application) and uses spectral similarity and spatial distance to weight contributions from known fine-resolution 
            pixels to predict reflectance for dates where only coarse imagery is available. 
            \item ESTARFM (Enhanced STARFM): \\
            An extension of STARFM designed to perform better in heterogeneous landscapes by considering spectral unmixing concepts implicitly 
            and using conversion coefficients derived from two fine/coarse image pairs before and after the desired prediction date. 
        \end{enumerate}
    \item Unmixing-based methods
        \begin{enumerate}
            \item STDFA (Spatio-Temporal Data Fusion Approach): \\
            Combines spectral unmixing and spatial interpolation to predict fine-resolution reflectance. 
        \end{enumerate}
    \item Regression-based methods
\end{itemize}

\subsubsection{Deep Learning STF algorithm candidates}
\begin{itemize}
    \item CNN-based methods
    \item GAN-based methods
\end{itemize}

\subsection{Implementation}
(add links to colab jupyter notebooks) \\
(add essential code excerpts to highlight algorithm details)

\subsection{Complexity Analysis}

\begin{itemize}
    \item CV factors: image size, number of bands, number of samples
    \item DL factors: number of layers, batch size, number of epochs, fp weight size
    \item non-DL factors: window size, other parameters 
    \item Big O notation
    \item Running time 
    \item Memory usage
\end{itemize}

\end{multicols}

\clearpage